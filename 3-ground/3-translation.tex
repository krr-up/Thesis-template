\subsection{Translating Temporal Theories}

Before we present our translation, we show that when we restrict
ourselves to finite traces, the $\release/\trigger$ operator can be
defined in terms of the $\until/\since$ operator, respectively
(J. Romero, personal communication 2023). This fact is relevant to our
discussion, as it allows us to simplify the translation to be
presented shortly by using the replacement property, and allows us to
later avoid having to implement the (arguably) less straightforward
semantics of $\release/\trigger$.

\begin{proposition}[Definability of $\release/\trigger$ via $\until/\since$]

If $\lambda \neq \infty$ then:
\begin{align*}
\varphi \release \psi &\equivtht \psi \until (\psi \wedge (\varphi \vee \finally)) \\
\varphi \trigger \psi &\equivtht \psi \since (\psi \wedge (\varphi \vee \initially))
\end{align*}
\end{proposition}
\begin{proof}
  We will show the first equivalence, as the second can be proven
  similarly. To prove strong equivalence in THT, we need to show that
  any $\kinlambda$ and HT trace $\thandt$,
  $\thandt,k \models \varphi \release \psi$ iff
  $\thandt,k \models \psi \until (\psi \wedge (\varphi \vee
  \finally)).$

  For the left to right direction assume that
  $\thandt,k \models \varphi \release \psi$. Then, per the definition
  of the operator, either $\thandt,k \models \psi$ for all
  $\rangeco{j}{k}{\lambda}$, or there exists some
  $\rangeco{j}{k}{\lambda}$ such that $\handt,j \models \psi$ and
  $\handt,i \models \varphi$ for any $i \in \rangeco{i}{k}{j}$. In
  both cases,
  $\handt \models \until (\psi \wedge (\varphi \vee \finally))$, with
  the time point required by the until being $\lambda-1$ in the first
  case, and $j$ in the second. For the right to left direction, a
  similar argument can be made.
\end{proof}

Note that in the infinite case, the first equivalence does not hold as
in the infinite setting we have $\models \neg \finally$. In fact, it
was show by \cite{babodife20a} that $\release$ is not definable in
terms of a $\release$-free formula in the infinite case; though the
argument made in \cite[p. 20]{agcadipescscvi20a} to extend the result
to the finite case does not hold, as shown by the proposition above.

We now present the translation from temporal theories to logic
programs. The formalization of this translation is inspired by
\cite[p. 9]{capeva05a}, where the authors define a translation from
arbitrary propositional theories to logic programs; the proof of our
upcoming theorem will follow the same logical structure as that of
Theorem 2. in the aforementioned paper.


Given a temporal theory $\Gamma$ over signature \A, we
define a new signature $\A_L$ as:
$$
\A_L = \{ L_{\varphi}^k \mid \kinlambda, \varphi \in
\mathcal{L}_\A\setminus \{ \bot, \top \} \}
$$

For any time point $\kinlambda$ and any non-atomic formula $\gamma \in \mathcal{L}_A^T$ the \emph{definition} of $\gamma$ at time point $k$, written as $df^k(\gamma) \in \mathcal{L}_{A_L}$, corresponds to:

\begin{align*}
df^k(\gamma) \defeq \begin{cases}
  L^k_{\gamma} \leftrightarrow L_{\varphi}^k \otimes L_{\psi}^k 
  &\text{ if } \gamma = \varphi \otimes \psi, \otimes \in \{ \vee, \wedge, \rightarrow \}\\[2ex]
  L^k_\gamma \leftrightarrow \begin{cases} 
    L^{k+1}_{\varphi} &\text{ when } k < \lambda - 1\\
    \bot &\text{ when } k = \lambda - 1
    \end{cases}
  &\text{ if } \gamma = \Next \varphi \\[2ex]
  L^k_{\gamma} \leftrightarrow \begin{cases} 
    L^{k-1}_{\varphi} &\text{ when } 0 < k\\
    \bot &\text{ when } k = 0
    \end{cases}
  &\text{ if } \gamma = \previous \varphi \\[2ex]
  L^k_{\gamma} \leftrightarrow \bigvee_{j=k}^{\lambda-1}(L_\psi^j \wedge \bigwedge_{i=k+1}^{j}L_{\varphi}^i)
  &\text{ if } \gamma = \varphi \until \psi \\[2ex]
  L^k_{\gamma} \leftrightarrow \bigvee_{j=0}^{k}(L_\psi^j \wedge \bigwedge_{i=j+1}^{k}L_{\varphi}^i)
  &\text{ if } \gamma = \varphi \since \psi \\[2ex]
\end{cases}
\end{align*}

Note that this implicitly also defines $df^k$ for all of our derived
operators. The definitions above, in essence, are a translation of the
satisfaction relations for THT, as described in the meta-language of
this text, into a concrete representation using propositional
connectives.

Given a temporal theory $\Gamma$, let $subf(\Gamma)$ denote all
subformulas occurring in $\Gamma$. We can now define the translation
of $\Gamma$ as:
$$
\chi(\Gamma) = \{ L_\varphi^0 \mid \varphi \in \Gamma \} 
\cup \{ df^k(\gamma) \mid \gamma \in subf(\Gamma), \kinlambda \}
$$

The first element of the union adds literals $L_\varphi^0$ who's
intended meaning is that $\varphi$ must hold at time point $0$ for any
$\varphi \in \Gamma$. The second element of the union realizes this
intention by recursively adding the definitions of all subformulas of
$\Gamma$ for all time points $\kinlambda$.

We would like to use the stable models of this translated theory to
find the temporal stable models of the input theory. It stands to
reason then, that some correspondence between the two must be
proven. To this end, let us define for any 3-valued interpretation $I$
over $\A_L$ a corresponding 3-valued interpretation $m_I$ over $\A$ as
$m_I(k,a) \defeq I(L_a^k)$ for any $a \in A$. Recall from Proposition
\ref{prop:3-valued-ht} that $I \models \varphi \leftrightarrow \psi$
iff $I(\varphi) = I(\psi)$.

\begin{theorem}\label{theorem:translation}
For any temporal theory $\Gamma$ over $\A$: \quad
$\{ m \mid m \models \Gamma \} = \{ m_I \mid I \models \chi(\Gamma) \}$
\end{theorem}
\begin{proof}
  $\subseteq$: Let $m$ be a 3-valued interpretation such that
  $m \models \Gamma$. We define a corresponding 2-valued
  interpretation $I$ as $I(L^k_\varphi) \defeq m(k,\varphi)$ for any
  formula $\varphi \in \mathcal{L}_\A$. Then, for this particular
  2-valued interpretation we have $m_I=I(L_a^k)=m$, so we have
  succeeded if we prove that $I \models \chi(\Gamma)$. Given that
  $I \models L^0_\gamma$ for any $\gamma \in \Gamma$, as
  $m \models \Gamma$, we have reduced the problem to showing that
  $I \models \{ df^k(\gamma) \mid \gamma \in subf(\Gamma), \kinlambda
  \}$.

  Recall from Proposition \ref{prop:3-valued-ht} that
  $I \models \varphi \leftrightarrow \psi$ iff $I(\varphi) =
  I(\psi)$. For a binary propositional connective
  $\otimes \in \{ \vee, \wedge, \rightarrow \}$ let $f_{\otimes}$,
  $f_{\otimes}^k$ denote it's associated 3 valued evaluation functions
  in the non-temporal and temporal setting, respectively. Then, we
  have
  $I \models L^k_{\varphi \otimes \psi} \leftrightarrow L^k_\varphi
  \otimes L^k_\psi$, as:
  $$
  I(L^k_{\varphi \otimes \psi}) \defeq m(k,\varphi \otimes \psi) 
  = f_{\otimes}^k(m(k,\varphi),m(k,\psi)) = f_{\otimes}(I(L^k_\varphi),I(L^k_\psi)) = I(L^k_{\varphi} \otimes L^k_{\psi})
  $$
  For $\gamma=\previous \varphi$, we have
  $$
  I(L^k_{\previous \varphi}) \defeq m(k,\previous \varphi) = \begin{cases}
    m(k-1,\varphi)=I(L^{k-1}_\varphi)=I(L^{k-1}_\varphi) &\text{ if } k > 0\\
    0=I(\bot) &\text{ if } k = 0
    \end{cases}
  $$
  and for $\gamma=\varphi \since \psi$ we have
  \begin{align*}
  I(L^k_{\varphi \since \psi}) \defeq m(k,\varphi \since \psi) &= \max \{\min (m(j, \psi), \min \{m(i, \varphi) \mid j<i \leq k\}) \mid 0 \leq j \leq k\} \\
    &= I(\bigvee_{j=0}^{k}(L_\psi^j \wedge \bigwedge_{i=j+1}^{k}L_{\varphi}^i))
  \end{align*}
  A similar argument can also be made for the definitions next and
  until, which concludes the proof of this direction.

  $\supseteq:$ To prove this direction, it suffices to show that for
  any interpretation $I$, if $I \models \chi(\Gamma)$, then
  $m_I \models \Gamma$. We will show by structural induction that if
  $\gamma \in subf(\Gamma)$, then $m_I(k,\gamma) = I(L^k_\gamma)$.
  Given this result, we will be able to conclude that
  $m_I \models \Gamma$, as by our initial assumption $I(L^0_\gamma)=2$
  for any $\gamma \in \Gamma$, and thus
  $m_I(0,\gamma)=I(L^0_\gamma)=2$.

  To prove the structural induction, note that for any atomic formula
  $\gamma \in \A \cup \{ \bot, \top \}$,
  $m_I(k,\gamma) = I(L^k_\gamma)$ does indeed hold per definition of
  $m_I$. Then, the following equalities prove the inductional step
  (again omitting future operators for brevity):
  \begin{align*}
  I(L^k_{\varphi \otimes \psi}) &\stackrel{*}{=} I(L^k_{\varphi} \otimes L^k_{\psi})
  =  f_{\otimes}(I(L^k_\varphi),I(L^k_\psi))\\
  &\stackrel{**}{=} f_{\otimes}^k(m(k,\varphi),m(k,\psi)) =  m_I(k,\varphi \otimes \psi)\\
  I(L^k_{\previous \varphi}) &\stackrel{*}{=} \left. \begin{cases}
    I(L^{k-1}_\varphi)  \stackrel{**}{=} m(k-1,\varphi) \text{ if } k > 0\\
    0 \text{ if } k = 0
    \end{cases} \right\} = m_I(k,\previous \varphi)\\
  I(L^k_{\varphi \since \psi}) 
    & \stackrel{*}{=}  I(\bigvee_{j=0}^{k}(L_\psi^j \wedge \bigwedge_{i=j+1}^{k}L_{\varphi}^i)) \\
    & \stackrel{**}{=} \max \{\min (m(j, \psi), \min \{m(i, \varphi) \mid j<i \leq k\}) \mid 0 \leq j \leq k\} \\
    & =  m(k,\varphi \since \psi)
  \end{align*}

  where in equalities $\stackrel{*}{=}$ we use th fact that $I \models
  \{ df^k(\gamma) \mid \gamma \in subf(\Gamma), \kinlambda
  \}$ and in equalities
  $\stackrel{**}{=}$ we use the induction hypothesis.
\end{proof}

\begin{corollary}\label{cor:bijection-of-sm}
  Let $\Gamma$ be a temporal theory and $\chi(\Gamma)$ it's
  translation. For any set of atoms $T \subset \A_L$ define the
  trace $\tau(T)=(T_k)_{\kinlambda}$, where
  $T_k = \{ a \in A \mid L_a^k \in X \}$. Then, $T$ is a stable model of
  $\chi(\Gamma)$ iff $\tau(T)$ is a temporal stable model of
  $\Gamma$.
\end{corollary}

Corollary \ref{cor:bijection-of-sm} thus establishes a method of
finding temporal stable models of a temporal theory by translating it
to a regular theory and finding their stable models. However, most
commonly used answer set solvers do not accept arbitrary theories as
input, but instead restrict the input to logic programs. Fortunately,
we can apply a set of strongly equivalent vocabulary-preserving
transformations described in \cite{capeva05a} to reduce the theory to
a logic program. It is also worthwhile in one case to apply the
non-vocabulary preserving transformation described in
\cite{capeva05a}. To to avoid the blowup in the number of rules
generated when unfolding the definitions of $\until$ and $\since$ via
the vocabulary-preserving transformation, we introduce auxiliary
atoms, and modify the definition of the operators to the following:
$L_{\gamma,j}^k$:
\begin{align*}
df^k(\gamma) \defeq \begin{cases}
  L^k_{\gamma} \leftrightarrow \bigvee_{j=k}^{\lambda-1}L_{\gamma,j}^k \wedge L_{\gamma,j}^k \leftrightarrow 
(L_\psi^j \wedge \bigwedge_{i=k+1}^{j}L_{\varphi}^i)
  &\text{ if } \gamma = \varphi \until \psi \\[2ex]
  L^k_{\gamma} \leftrightarrow  \bigvee_{j=0}^{k} L_{\gamma,j}^k \wedge L_{\gamma,j}^k \leftrightarrow (L_\psi^j \wedge \bigwedge_{i=j+1}^{k}L_{\varphi}^i)
  &\text{ if } \gamma = \varphi \since \psi \\[2ex]
\end{cases}
\end{align*}