
\vspace*{\fill}
\thispagestyle{empty}
\bigskip
\begin{center}\bfseries{Abstract}
\end{center}
\noindent\makebox[\textwidth][c]{
    \begin{minipage}{0.9\textwidth}
    Temporal and dynamic extensions of Answer Set Programming (ASP) have played an important role addressing dynamic problems, such as planning, as they allow the use of temporal operators to reason with dynamic scenarios in a very effective way. 
    In this project, we exploit the relationship between automata theory and both linear temporal and linear dynamic logics on finite traces. 
    This relationship allows us to represent temporal constraints from ASP in terms of alternating automata.
    Our automata-based approach generates a declarative representation of the alternating automaton, which enables two different reasoning tasks: generating traces satisfying a constraint and checking the satisfiability of a given trace. 
    Regarding practical applications, the implementation was tested with intra-logistics among multiple robots, where it provided a concise way of filtering plans that enforce temporal goals.
    Eventoght the performance results were not favorable, this work remains a profitable first survey of these connection, as it helped define the path for future research. 
\end{minipage}}
\bigskip
\begin{center}\bfseries{\textit{\textcolor{darkblue}{Kurzfassung}}}
\end{center}
\noindent\makebox[\textwidth][c]{
    \begin{minipage}{0.9\textwidth}
        \textit{\textcolor{darkblue}{
Zeitliche und dynamische Erweiterungen der Answer Set Programming (ASP) haben bei der Bewältigung dynamischer Probleme, wie z.B. der Planung, eine wichtige Rolle gespielt, da sie die Verwendung von zeitlichen Operatoren ermöglichen, um mit dynamischen Szenarien auf sehr effektive Weise zu argumentieren. 
In diesem Projekt nutzen wir die Beziehung zwischen der Automatentheorie und sowohl linearen zeitlichen als auch linearen dynamischen Logiken auf endlichen Spuren. 
Diese Beziehung erlaubt es uns, die zeitlichen Zwänge von ASP in Form von alternierenden Automaten darzustellen.
Unser auf Automaten basierender Ansatz erzeugt eine deklarative Darstellung des alternierenden Automaten, die zwei verschiedene Argumentationsaufgaben ermöglicht: die Erzeugung von Spuren, die eine Randbedingung erfüllen, und die Überprüfung der Erfüllbarkeit einer gegebenen Spur. 
Was die praktischen Anwendungen betrifft, so wurde die Implementierung mit Intralogistik zwischen mehreren Robotern getestet, wo sie eine prägnante Möglichkeit bietet, Pläne zu filtern, die zeitliche Ziele erzwingen.
Auch wenn die Leistungsergebnisse nicht günstig waren, bleibt diese Arbeit eine gewinnbringende erste Untersuchung dieser Zusammenhänge, da sie dazu beitrug, den Weg für zukünftige Forschungen zu definieren.}}
        
\end{minipage}}
\vspace*{\fill}