\subsection{THT and TEL}

Linear time temporal logic (LTL) enhances propositional logic with a
linear sequence of worlds and temporal modal operators which allows us
to express temporal conditions over these sequence of worlds, such as
a formula holding in all future time points, or a formula eventually
holding in some future time point. To define a non-monotonic version
of LTL, we take similar approach as in propositional logic, defining
first a monotonic Here-and-There variant, among which we define models
which are in equilibrium via a totality and minimality condition. The
definitions and propositions of this section are taken from
\cite{agcadipescscvi20a}.

A temporal formula over a set of atoms $\A$ is defined by the
following grammar:

\begin{align*}
    \varphi ::= &\; a \mid \bot \mid
                  \varphi_1 \otimes \varphi_2 \mid
                  \previous \varphi \mid \varphi_1 \since \varphi_2 \mid \varphi_1 \trigger \varphi_2 \mid
                  \Next \varphi \mid \varphi_1 \until \varphi_2 \mid \varphi_1 \release \varphi_2 \mid                  
\end{align*}
where $\otimes \in \{ \wedge, \vee, \to \}$ is a binary Boolean
connective. We denote set of all temporal formulas as the language
$\mathcal{L}_{A}^{T}$, and a subset of $\mathcal{L}_{A}^{T}$ as a
temporal theory. We sometimes drop the temporal adjective from the
definitions of this subsection when it is clear from context that we
are talking about the temporal setting. The temporal operators
$\previous, \since, \trigger, \Next, \until$ and $\release$ are read as
previous, since, trigger, next, until and release,
respectively. Additionally, we define following derived temporal
operators.

\[
\begin{array}{l l l l l l l l}
\alwaysP \varphi & \defeq & \bot \trigger \varphi & \text{\emph{always before}} &
\alwaysF \varphi & \defeq & \bot \release \varphi & \text{\emph{always after}} \\
\eventuallyP \varphi & \defeq & \top \since \varphi & \text{\emph{eventually before}} &
\eventuallyF \varphi & \defeq & \top \until \varphi & \text{\emph{eventually after}} \\
\initially & \defeq & \lnot \previous \top & \text{\emph{initial}} &
\finally & \defeq & \lnot \Next \top & \text{\emph{final}} \\
\wprevious & \defeq & \Next \varphi \vee \initially & \text{\emph{weak previous}} &
\wnext & \defeq & \Next \varphi \vee \finally & \text{\emph{weak next}}
\end{array}
\]

The temporal above can be split into groups based on whether they act
on past or future time points, and can be distinguished visually; past
operators $\previous, \since, \trigger$ are filled, while future
operators $\Next, \until, \release$ are outlined.

To give an informal summary of the semantics of the operators, the
formulas $\previous \varphi$/$\Next \varphi$ say that $\varphi$ is
true at previous/next time point, respectively.  $\varphi \since \psi$
says that $\varphi$ has been true since some time point in the past
where $\psi$ was true, while $\varphi \until \psi$ says that $\varphi$
must be true until some time point in the future where $\psi$ is true.
The meaning of trigger and release are somewhat less
intuitive. $\varphi \trigger \psi$ mean that $\psi$ has been true
either for all time steps in the past, or since some time point in the
past when $\varphi$ and $\psi$ were both true. $\varphi \release \psi$
in turn means that $\psi$ will be true either for all time steps in
the future, or until some time point in the future when $\varphi$ and
$\psi$ will both be true.

To formalize the notion of a time point, let us first fix some
terminology and notation. We define the extended natural numbers
$\mathbb{N}^\infty$ to be the set of natural numbers $\mathbb{N}$
extended with a maximum element $\infty$, i.e.
$\mathbb{N}^\infty = \mathbb{N} \cup \{ \infty \}$. For any
$a,b \in \mathbb{N}^\infty$, we define the following integer intervals
as the sets $\intervcc{a}{b}=\{ i \in \mathbb{N} \mid a \leq i \leq b \}$,
$\intervco{a}{b}=\{ i \in \mathbb{N} \mid a \leq i < b \}$. A trace of length
$\bm{T}$ over signature $\A$ of length
$\lambda \in \mathbb{N}^\infty$ is then defined as a (possibly
infinite) sequence of sets of atoms $\bm{T} = (T_i)_{\rangeco{i}{0}{\lambda}}$
where $T_i \subseteq \A$.

If for each time point we consider two worlds, a "here" and a "there",
instead of a single world, we arrive at the notion of an HT
trace. Formally, an HT trace $\thandt$ of length $\lambda$ is an
ordered pair of traces, where for each $\rangeco{i}{0}{\lambda}$
$H_i \subseteq T_i$. An $HT$ trace is total iff $\bm{H} = \bm{T}$. We
are now ready to extend the logic of HT with the temporal operators
defined above, resulting in the logic of Temporal Here-And-There, or
THT for short.

\begin{definition}[THT satisfaction \cite{agcadipescscvi20a}]
  An HT-trace $\thandt$ of length $\lambda$ over signature $\A$
  satisfies a temporal formula $\varphi$ at time point
  $\rangeco{k}{0}{\lambda}$, written as $\thandt,k \models \varphi$ if
  the following condition holds, where we use the shorthand
  $\bf{M} =$ $\thandt$.
\begin{description}
  \item $\bf{M},k \models \top$ and $\bf{M}, k \not \models \perp$
  \item $\bf{M},k \models a$ if $a \in H_{k}$ for any atom $a \in \mathcal{A}$
  \item $\bf{M},k \models \varphi \wedge \psi$ iff $\bf{M}, k \models \varphi$ and $\bf{M}, k \models \psi$
  \item $\bf{M},k \models \varphi \vee \psi$ iff $\bf{M}, k \models \varphi$ or $\bf{M}, k \models \psi$
  \item $\bf{M},k \models \varphi \rightarrow \psi$ iff 
    $\langle \bm{H}^{\prime},\bm{T} \rangle, k \not \models \varphi$ 
    or $\langle \bm{H}^{\prime},\bm{T} \rangle, k \models \psi$, 
    for all $\bm{H}^{\prime} \in\{\bm{H}, \bm{T}\}$
  \item $\bf{M},k \models \previous \varphi$ iff $k>0$ and $\bf{M},k-1 \models \varphi$
  \item $\bf{M},k \models \varphi \since \psi$ iff for some $j \in[0 . . k]$, we have $\bf{M}, j \models \psi$ and $\bf{M},i \models \varphi$ for all $i \in(j . . k]$
  \item $\bf{M},k=\varphi \trigger \psi$ iff for all $j \in[0 . . k]$, we have $\bf{M}, j \models \psi$ or $\bf{M}, i \models \varphi$ for some $i \in(j . . k]$
  \item $\bf{M},k \models \Next \varphi$ iff $k+1<\lambda$ and $\bf{M}, k+1 \models \varphi$
  \item $\bf{M}, k \models \varphi \until \psi$ iff for some $j \in[k . . \lambda)$, we have $\bf{M}, j \models \psi$ and $\bf{M}, i \models \varphi$ for all $i \in[k . . j)$
  \item $\bf{M}, k=\varphi \release \psi$ iff for all $j \in[k
    . . \lambda)$, we have $\bf{M}, j \models \psi$ or $\bf{M},
    i=\varphi$ for some $i \in[k . . j)$
\end{description}
\end{definition}

The satisfaction conditions for derived temporal operators can be
inferred from the above definitions; for more details we refer the
reader to \cite{agcadipescscvi20a}.

We say that an HT-trace $\thandt$ is a THT model of a temporal theory
$\Gamma$ iff $\thandt,0 \models \varphi$ for all $\varphi \in
\Gamma$. A formula $\varphi$ is a THT tautology, written as
$\models \varphi$, if $\handt,k \models \varphi$ for any HT-trace
$\thandt$ and $\rangeco{k}{0}{\lambda}$. We call the logic induced by
the set of all tautologies the logic of Temporal Here-and-There (THT
for short). Two formulas $\varphi$ and $\psi$ are said to be THT
equivalent, denoted as $\varphi \equivtht \psi$, iff
$\models \varphi \leftrightarrow \psi$. Note that while THT models are
only required to satisfy a formula at the initial time step, to verify
that a formula is a tautology, one must check not only for any
HT-trace, but also any time point that the equivalence holds; this
applies to strong equivalence as well, as it is defined in terms of
tautology.

THT, similarly to HT, can also be characterized via a 3-valued THT
interpretation $m$. In the case of THT though, the interpretation is
also parameterized by the time point. For any time point $\kinlambda$,
$m(k,a)=2$ iff $a \in H_k$, $m(k,a)=1$ iff $a \in T_k \setminus H_k$
and $m(k,a)=0$ iff $a \not\in T_i$. The interpretation can be extended
to arbitrary temporal formulas using the rules below
\cite{agcadipescscvi20a}, where $\operatorname{imp}(x,y)=2$ if
$x \leq y$, otherwise $\operatorname{imp}(x,y)=y$.

\begin{align*}
  m(k, \perp) &\defeq 0 \\
  m(k, \varphi \wedge \psi) &\defeq \min (m(k, \varphi), m(k, \psi)) \\
  m(k, \varphi \vee \psi) &\defeq \max (m(k, \varphi), m(k, \psi)) \\
  m(k, \varphi \rightarrow \psi) &\defeq \operatorname{imp}(m(k, \varphi), m(k, \psi)) \\
  m(k, \previous \varphi) &\defeq \begin{cases}
    0 & \text { if } k=0 \\
    m(k-1, \varphi) & \text { if } k>0
  \end{cases} \\
 m(k, \varphi \since \psi) &\defeq \max \{\min (m(j, \psi), \min \{m(i, \varphi) \mid j<i \leq k\}) \mid 0 \leq j \leq k\} \\
 m(k, \varphi \trigger \psi) &\defeq \min \{\max (m(j, \psi), \max \{m(i, \varphi) \mid j<i \leq k\}) \mid 0 \leq j \leq k\} \\
 m(k, \Next \varphi) &\defeq \begin{cases}0 & \text { if } k+1=\lambda(\neq \omega) \\
m(k+1, \varphi) & \text { if } k+1<\lambda\end{cases} \\
 m(k, \varphi \until \psi) &\defeq \max \{\min (m(j, \psi), \min \{m(i, \varphi) \mid k \leq i<j\}) \mid k \leq j<\lambda\} \\
 m(k, \varphi \release \psi) &\defeq \min \{\max (m(j, \psi), \max \{m(i, \varphi) \mid k \leq i<j\}) \mid k \leq j<\lambda\}
\end{align*}

The 3-valued THT interpretation satisfies similar properties as the
3-valued HT interpretation discussed in the previous section.

\begin{proposition}[\cite{capeva05a}]
  For any 3-valued THT interpretation $m$, $\kinlambda$ and temporal
  formulas $\varphi$, $\psi$:
\begin{description}
  \item $\thandt,k \models \varphi$ iff $m(k,\varphi) = 2$
  \item $\ttandt,k \models \varphi$ iff $(\varphi) \neq 0$
  \item $\thandt,k \models \varphi \leftrightarrow \psi$ iff $m(k,\varphi) = m(k,\psi)$
\end{description}
\end{proposition}

We write $m \models \varphi$ iff $m(0,\varphi)=2$.

Temporal equilibrium models and temporal stable models are defined
similarly as in the HT case.

\begin{definition}[Temporal Equilibrium Model/Temporal Stable Model \cite{agcadipescscvi20a}]
  A total HT-trace $\ttandt$ is said to be a temporal equilibrium
  model of a temporal theory $\Gamma$ iff $\ttandt \models \Gamma$,
  and there is no HT-trace $\thandt$ such that $\thandt \models \Gamma$ and
  $\bm{H} \subset \bm{T}$. A trace $\bm{T}$ is a temporal stable model of a temporal theory $\Gamma$ iff
  $\tandt$ is a temporal equilibrium model of $\Gamma$.
\end{definition}

Furthermore, similarly to HT, strong equivalence of two formulas in
THT allows us to replace one formula with another within a theory
without affecting the temporal stable models of the theory as a whole

\begin{proposition}[Strong Equivalence and THT Equivalence \cite{agcadipescscvi20a}]
  Two temporal theories $\Gamma_1$ and $\Gamma_2$ are strongly equivalent
  iff $\Gamma_1 \equivtht \Gamma_2$
\end{proposition}

\begin{proposition}[Replacement property \cite{agcadipescscvi20a}]
  Let $\gamma[\varphi]$ denote a temporal formula with some occurrence of a
  subformula $\varphi$ and let $\gamma[\psi]$ be the temporal formula resulting
  from replacing said occurrence of $\varphi$ with $\psi$. If
  $\varphi \equivtht \psi$, then $\gamma[\varphi] \equivtht \gamma[\psi]$.
\end{proposition}