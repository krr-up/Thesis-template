%%%%%%%%%%%%%%%%%%%%%%%%%%%%%%%%%%%%%%%%%%%%%%%%
%                 Temporal into Automaton
%%%%%%%%%%%%%%%%%%%%%%%%%%%%%%%%%%%%%%%%%%%%%%%%
\subsection{From temporal formulas to alternating automata}

%%%%%%%%%%%%%%%%% LTLf %%%%%%%%%%%%%%%%%%%%%%%%%
\subsubsection{$\LTLf$ into alternating automata}


Given a temporal formula $\varphi$ in \nnf, we can build an $\AFW$  $A_\varphi$, such that the state-size of $A_\varphi$ is linear in the size of $\varphi$, and $L(A_\varphi)$ is exactly the traces that satisfy $\varphi$. In this translation, finite traces are viewed as finite words over the alphabet of truth assignments to the atomic proposition.

As a first step in the translation, the states of the automaton are constructed using the concept of the \emph{closure of a formula} $\varphi$.

\begin{definition}[Closure]
    The closure of a temporal formula $\varphi$, written as $cl(\varphi)$ is a the smallest set defined as follows:
    \begin{itemize}
        \item $\varphi \in cl(\varphi)$;
        % \item $\neg \psi \in cl(\varphi)$ if $\psi \in cl(\varphi)$ and is not of the form $\neg \gamma$
        \item if $\otimes \psi \in cl(\varphi)$ then $\psi \in cl(\varphi)$ for $\oplus \in \lbrace \Next, \wnext \rbrace$
        \item if $\gamma \oplus \psi \in cl(\varphi)$ then both $\gamma\in cl(\varphi)$ and $\psi \in cl(\varphi)$ for $\oplus \in \lbrace \wedge, \vee, \until, \release, \rbrace$
        %\item if $\gamma \rightarrow \psi \in cl(\varphi)$ if both $\neg \gamma, \psi \in cl(\varphi)$
        \item if $\gamma \until \psi \in cl(\varphi)$ then $\Next\left( \gamma \until \psi\right) \in cl(\varphi)$
        \item if $\gamma \release \psi \in cl(\varphi)$ then  $\wnext\left( \gamma \release \psi\right) \in cl(\varphi)$
    \end{itemize}
\end{definition}

Using this definition, we consider all quoted formulas $``\psi"$ where $\psi \in cl(\varphi)$  as states\footnote{Quotes are used to better distinguish states from formulas.}.

For the alphabet of the automaton, we take into account the end of the trace by including a special proposition $\last$, which is only true at the last time point of the trace. This proposition allows a case analysis in the transition of operators $\Next$ and $\wnext$ based on the interpretation in the current step of the trace. By adding it to the set of atomic propositions $\A$, we construct the alphabet $\Sigma=2^{\A \cup \{\last\}}$, thus representing a finite trace $\pi$ as a finite word $w$ over $\Sigma$, where each state of the trace is a propositional interpretation $\Pi \in \Sigma$.

\begin{definition}[$\LTLf$ to $\AFW$]
    The $\AFW$ for a temporal formula $\varphi$ in \nnf is defined as:
    $A_\varphi =(2^{\A \cup \{\last\}},\{``\psi" \mid \psi \in cl(\varphi) \},``\varphi",\delta,\emptyset)$ where the transition function $\delta = S\times \Sigma \to B^+(S)$ from a quoted formula and an interpretation $\Pi\in\Sigma$ is defined as follows: 


\begin{multicols}{2}
    \begin{itemize}
        \item $\delta(``\top",\Pi)= \mathit{\tpbf}$

        \item $\delta(``a",\Pi)=\begin{cases}
          \bm{true} & \text{if} \;a\in \Pi \\
          \fpbf & \text{if} \;a\not\in \Pi \\
        \end{cases}$
        
        \item $\delta(``\varphi_1 \vee \varphi_2",\Pi)=\delta(``\varphi_1",\Pi) \vee $\\\phantom{$\delta(``\varphi_1 \vee \varphi_2\;",\Pi)= $}$\delta(``\varphi_2",\Pi)$

        
        \item $\delta(``\Next \varphi",\Pi)=\begin{cases}
          ``\varphi" & \text{if} \;\last\not\in \Pi \\
          \fpbf & \text{if} \;\last\in \Pi \\
        \end{cases}$
        
%%%%%%%%%
        \item $\delta(``\bot",\Pi)= \fpbf$
        

        \item $\delta(``\neg a",\Pi)=\begin{cases}
            \fpbf & \text{if} \;a\in \Pi \\
            \bm{true} & \text{if} \;a\not\in \Pi \\
          \end{cases}$

                  
        \item $\delta(``\varphi_1 \wedge \varphi_2",\Pi)=\delta(``\varphi_1",\Pi) \wedge $\\\phantom{$\delta(``\varphi_1 \wedge \varphi_2\;",\Pi)= $}$\delta(``\varphi_2",\Pi)$
        
        \item $\delta(``\wnext \varphi",\Pi)=\begin{cases}
            ``\varphi" & \text{if} \;\last\not\in \Pi \\
            \bm{true} & \text{if} \;\last\in \Pi \\
          \end{cases}$
        % \item $\delta(``\varphi_1 \release \varphi_2",\Pi)=\delta(``\varphi_2",\Pi) \wedge $\\$ (\delta(``\varphi_1",\Pi) \vee \delta(``\wnext \varphi_1 \release \varphi_2",\Pi))$
      \end{itemize}
    \end{multicols}

    \begin{itemize}

        \item $\delta(``\varphi_1 \until \varphi_2",\Pi)=\delta(``\varphi_2",\Pi) \vee (\delta(``\varphi_1",\Pi) \wedge \delta(``\Next \varphi_1 \until \varphi_2",\Pi))$
    
        \item $\delta(``\varphi_1 \release \varphi_2",\Pi)=\delta(``\varphi_2",\Pi) \wedge (\delta(``\varphi_1",\Pi) \vee \delta(``\wnext \varphi_1 \release \varphi_2",\Pi))$
        
        % Unfolded


        % \item $\delta(``\varphi_1 \until \varphi_2",\Pi)=\begin{cases}
        %     \delta(``\varphi_2",\Pi) & \text{if} \;\last\in \Pi \\
        %     \delta(``\varphi_2",\Pi) \vee (\delta(``\varphi_1",\Pi) \wedge `` \varphi_1 \until \varphi_2") & \text{if} \;\last\not\in \Pi \\
        %   \end{cases}$
        % \item $\delta(``\varphi_1 \release \varphi_2",\Pi)=\begin{cases}
        % \delta(``\varphi_2",\Pi) & \text{if} \;\last\in \Pi \\
        % \delta(``\varphi_2",\Pi) \wedge (\delta(``\varphi_1",\Pi) \vee `` \varphi_1 \release \varphi_2") & \text{if} \;\last\not\in \Pi \\
        % \end{cases}$
    \end{itemize}


\end{definition}


We introduce a \emph{transition diagram} to represent the case analysis of function $\delta$ wrt. $\Pi$, by considering a new connective $?$ such that restrictions over the interpretation only occur after this connective and edges without label apply to any interpretation. 

To simplify the notation in our examples, we use $\interp{x}$ to represent all interpretations $\Pi$ such that $x\in\Pi$ and $\interp{\neg x}$ for all $\Pi$ where $x\not\in\Pi$.
%------------------ Example Na and b unitl a --------

\begin{example}[Transition diagram]
    Given the formula $\varphi= \Next a \wedge (b \until a)$ already in nnf. The AFW$_f$ automaton for $\varphi$ is $A_\varphi =(2^{\{a,b,\last\}},S,\s{\varphi},\delta,\emptyset)$ where $S=\{\s{\varphi},\s{\Next a},\s{b \until a},\s{a},\s{b}\}$. The transition $\delta$ is represented in the following transition diagram. States marked in blue are not reachable from the initial state, however, they are still computed as part of the inductive definition.

    \begin{tikzpicture}[shorten >=1pt,node distance=1.7cm,on grid,auto]
    \tikzstyle{every state}=[fill={rgb:black,1;white,20},font=\small]
    \tikzstyle{every pin edge}=[->,shorten <=1pt]
    \tikzstyle{interp}=[font=\footnotesize]
    \tikzstyle{conn}=[rectangle,draw,font=\small]
    \tikzstyle{bool}=[draw,font=\small]
    
    \node[state,initial]  (s_0)                 {$\varphi$};
    \node[conn]                    (and_1) [right of=s_0]  {$\wedge$};
    \node[conn]                    (check_1) [below left of=and_1]  {$?$};
    \node[conn]                    (or_1) [below right of=and_1]  {$\vee$};

    \node (false_1) [below left of=check_1] {\fpbf};
    \node[state,fill=darkblue!20]                    (next_a) [left of=check_1]  {$\Next a$};
    \node[state]                    (s_a) [below right of=check_1]  {$a$};
    \node[state,fill=darkblue!20] (s_b) [right of=s_a] {$b$};

    \node[conn]                    (and_2) [right of=s_b]  {$\wedge$};

    \node[conn]                    (check_a) [below of=s_a]  {$?$};
    \node[conn]                    (check_b) [below of=s_b]  {$?$};
    \node[conn]                    (check_l) [below right of=and_2]  {$?$};

    \node (true_b) [below right of=check_a] {\tpbf};

    \node (false_b) [below right of=check_b] {\fpbf};
    \node (false_a) [below left of=check_a] {\fpbf};
    \node (false_l) [below of=check_l] {\fpbf};

    \node[state] (s_until) [above of=or_1] {$a \until b$};
    


  
    \path[->]
    (s_0) edge  (and_1)
    (and_1) edge (check_1)
    (next_a) edge (check_1)
    (and_1) edge (or_1)
    (check_1) edge node[interp] {$\interp{\last}$} (false_1)
    (check_1) edge node[interp] {$\interp{\neg \last}$} (s_a)
    (or_1) edge (check_a)
    (or_1) edge (and_2)
    (s_a) edge (check_a)
    (s_b) edge (check_b)
    (check_a) edge node[interp] {$\interp{\neg a}$} (false_a)
    (check_a) edge node[interp] {$\interp{a}$} (true_b)
    (check_b) edge node[interp] {$\interp{b}$} (true_b)
    (check_b) edge node[interp] {$\interp{\neg b}$} (false_b)
    (and_2) edge (check_l)
    (and_2) edge (check_b)
    (check_l) edge node[interp] {$\interp{\last}$}(false_l)
    (check_l) edge[bend right=50]  node[interp] {$\interp{ \neg \last}$} (s_until)       
    (s_until) edge (or_1)
    ;
\end{tikzpicture}


\end{example}

To compute the alternating automata we must consider restrictions on the interpretation at the beginning of the transitions in order to follow the syntax of the function $\delta$. This process is performed by expanding each case and exploiting Boolean equivalences, as shown in the following example.

%------------------ Example Na and Nb or Nc --------
\begin{example}[Translation with branching]
    Given the formula $\varphi=\wnext a \wedge (\Next b\vee\Next c)$, already in \textit{nnf}, the AFW$_f$ automaton for $\varphi$ is $A_\varphi =(2^{\{a,b,c,\last\}},S,``\varphi",\delta,\emptyset)$ where $S=\{``\varphi",``\wnext a",``(\Next b)\vee(\Next c)",`\Next b`",``\Next c",``a",``b",``c"\}$. The corresponding transition diagram is showed below. 

    \vspace{5px}
    \begin{tikzpicture}[shorten >=1pt,node distance=1.7cm,on grid,auto]
    \tikzstyle{every state}=[fill={rgb:black,1;white,20},font=\small]
    \tikzstyle{every pin edge}=[->,shorten <=1pt]
    \tikzstyle{interp}=[font=\footnotesize]
    \tikzstyle{conn}=[rectangle,draw,font=\small]
    \tikzstyle{bool}=[draw,font=\small]

    \node[state,initial]  (s_0)                 {$\varphi$};
    \node[conn]   (and) [right of=s_0]  {$\wedge$};
    \node[conn]   (or) [right of=and]  {$\vee$};
    \node[state,fill=darkblue!20]  (s_or) [right of=or]  {$\Next b \vee \Next c$};
    \node   (e_2) [right of=s_or]  {};
    
    \node[state,fill=darkblue!20]                    (s_wa) [below of=s_0]  {$\wnext a$};
    \node[conn]   (check_l_a) [below of=s_wa]  {$?$};
    \node (true_l_a) [below left of=check_l_a] {\tpbf};
    \node[state] (s_a) [below right of=check_l_a] {$a$};
    \node[conn]   (check_a) [below of=s_a]  {$?$};
    \node (false_a) [below left of=check_a] {$\fpbf$};
    \node (true_a) [below right of=check_a] {\tpbf};

    \node[state,fill=darkblue!20]                    (s_nb) [below of=or]  {$\Next b$};
    \node[conn]   (check_l_b) [below of=s_nb]  {$?$};
    \node (true_l_b) [below left of=check_l_b] {$\fpbf$};
    \node[state] (s_b) [below right of=check_l_b] {$b$};
    \node[conn]   (check_b) [below of=s_b]  {$?$};
    \node (false_b) [below left of=check_b] {$\fpbf$};
    \node (true_b) [below right of=check_b] {\tpbf};

    \node[state,fill=darkblue!20]                    (s_nc) [below of=e_2]  {$\Next c$};
    \node[conn]   (check_l_c) [below of=s_nc]  {$?$};
    \node (true_l_c) [below left of=check_l_c] {$\fpbf$};
    \node[state] (s_c) [below right of=check_l_c] {$a$};
    \node[conn]   (check_c) [below of=s_c]  {$?$};
    \node (false_c) [below left of=check_c] {$\fpbf$};
    \node (true_c) [below right of=check_c] {\tpbf};


    \path[->]
    (s_0) edge  (and)
    (s_wa) edge  (check_l_a)
    (s_nb) edge  (check_l_b)
    (s_nc) edge  (check_l_c)
    (and) edge  (check_l_a)
    (and) edge  (or)
    (s_or) edge  (or)
    (or) edge[bend right=30]  (check_l_b)
    (or) edge  (check_l_c)
    (check_l_a) edge node[interp] {$\interp{ \last }$} (true_l_a)
              edge node[interp] {$\interp{ \neg \last }$} (s_a)
    (s_a) edge  (check_a)
    (check_a) edge node[interp] {$\interp{ \neg a }$} (false_a)
          edge node[interp] {$\interp{ a }$} (true_a)

    (check_l_b) edge node[interp] {$\interp{ \last }$} (true_l_b)
          edge node[interp] {$\interp{ \neg \last }$} (s_b)
(s_b) edge  (check_b)
(check_b) edge node[interp] {$\interp{ \neg b }$} (false_b)
      edge node[interp] {$\interp{ b }$} (true_b)


      (check_l_c) edge node[interp] {$\interp{ \last }$} (true_l_c)
              edge node[interp] {$\interp{ \neg \last }$} (s_c)
    (s_c) edge  (check_c)
    (check_c) edge node[interp] {$\interp{ \neg c }$} (false_c)
          edge node[interp] {$\interp{ c }$} (true_c)
    ;
\end{tikzpicture}

    The first cases of the transition function $\delta$ follow directly from the definition without further expansion. This type of base cases are considered as trivial and omitted in the following examples.

    \begin{multicols}{2}
        \small
        \begin{itemize}
            \item $\delta(``a",\interp{a })=\bm{true}$
            \item $\delta(``a",\interp{\neg a })=\bm{\mathit{false}}$
            \item $\delta(``b",\interp{b })=\bm{true}$
            \item $\delta(``b",\interp{\neg b })=\bm{\mathit{false}}$
            \item $\delta(``c",\interp{c })=\bm{true}$
            \item $\delta(``c",\interp{\neg c })=\bm{\mathit{false}}$
            \item $\delta(``\wnext a",\interp{\last })=\bm{\mathit{true}}$
            \item $\delta(``\wnext a",\interp{\neg \last })=``a"$
            \item $\delta(``\Next b",\interp{\last })=\bm{\mathit{false}}$
            \item $\delta(``\Next b",\interp{\neg \last })=``b"$
            \item $\delta(``\Next c",\interp{\last })=\bm{\mathit{false}}$
            \item $\delta(``\Next c",\interp{\neg \last })=``c"$
        \end{itemize}
    \end{multicols}

    To compute the transition of states $``\Next b \vee \Next c"$ and $\s{\varphi}$, we must unfold the inductive definition for different cases.


    \begin{itemize}
        
        \small
        \item $\delta(``\Next b \vee \Next c",\interp{\last})= \delta(``b",\interp{\last}) \vee \delta(``c",\interp{\last})\\
        \phantom{\delta(``\Next b \vee \Next c",\interp{\last})} \equiv \fpbf \vee \fpbf \equiv \fpbf$

        \item $\delta(``\Next b \vee \Next c",\interp{\neg \last})= \delta(``b",\interp{\neg \last}) \vee \delta(``c",\interp{\neg \last}) \equiv ``b" \vee ``c"$
        
        \item $\delta(``\varphi",\interp{\neg \last})= \delta(``a",\interp{\neg \last}) \wedge \delta(``\Next b \vee \Next c",\interp{\neg \last}) \equiv ``a" \wedge (``b" \vee ``c")$
        
        \item $\delta(``\varphi",\interp{\last})= \delta(``a",\interp{\last}) \wedge \delta(``\Next b \vee \Next c",\interp{\last})\\
        \phantom{\delta(``\varphi",\interp{\last})} \equiv \fpbf \wedge \bm{true} \equiv \fpbf$
    \end{itemize}

    These definitions lead to the following alternating automaton, simplifying the transition system.

    \begin{tikzpicture}[shorten >=1pt,node distance=1.8cm,on grid,auto]
    \tikzstyle{every state}=[fill={rgb:black,1;white,20},font=\small]
    \tikzstyle{every pin edge}=[->,shorten <=1pt]
    \tikzstyle{interp}=[font=\footnotesize]
    \tikzstyle{conn}=[rectangle,draw,font=\small]
    \tikzstyle{bool}=[draw,font=\small]

    \node[state,initial]  (s_0)                 {$\varphi$};
    \node[conn]   (and) [right of=s_0]  {$\wedge$};
    \node[conn]   (or) [right of=and]  {$\vee$};

    \node[state] (s_a) [below of=and] {$a$};
    \node (false_a) [below left of=s_a] {$\fpbf$};
    \node (true_a) [below right of=s_a] {\tpbf};

    \node (false) [below of=s_0] {$\fpbf$};
    \node[state] (s_b) [right of=s_a] {$b$};
    \node (false_b) [below right of=s_b] {$\fpbf$};

    \node[state] (s_c) [right of=s_b] {$c$};
    \node (true_c) [below right of=s_c] {\tpbf}; 


    \path[->]
    (s_0) edge[bend left=10] node[interp] {$\interp{\neg \last }$}    (and)
    (and) edge  (s_a)
    (and) edge[bend left=10]  (or)
    (or) edge[bend right=10]  (s_b)
    (or) edge[bend left=10]  (s_c)

    (s_0) edge node[interp] {$\interp{\last }$}  (false)
    
    (s_a) edge[bend right=10] node[interp] {$\interp{\neg a }$} (false_a)
          edge[bend left=10] node[interp] {$\interp{a }$} (true_a)
    
    (s_b) edge[bend right=10] node[interp] {$\interp{b }$} (true_a)
          edge[bend left=10] node[interp] {$\interp{\neg b }$} (false_b)
    (s_c) edge[bend right=10] node[interp] {$\interp{\neg c }$} (false_b)
          edge[bend left=10] node[interp] {$\interp{c }$} (true_c)
    ;
\end{tikzpicture}


    The next figure shows the runs on automaton $A_{\varphi}$ for trace $\pi=\emptyset,\{a,b\}$. Only run $R_1$ is accepted, since runs $R_2$ and $R_3$ hit a false transition. The trace is therefore accepted by the automaton. 

    \begin{tikzpicture}[shorten >=1.5pt,node distance=1.5cm,on grid,auto]
    % \tikzstyle{every state}=[fill={rgb:black,1;white,20}]
    \tikzstyle{every pin edge}=[->,dotted,shorten <=1pt]
    % \tikzstyle{conn}=[rectangle,draw]
    % \tikzstyle{bool}=[rectangle,draw]
    \node[]  (time)                 {\textcolor{darkblue}{TIME}};
    \node[]  (r_1)[right of =time]               {\textcolor{darkblue}{$R_1$}};
    \node[]  (r_11)[right of =r_1]               {};
    \node[]  (r_2)[right of =r_11]               {\textcolor{darkred}{$R_2$}};
    \node[]  (r_21)[right of =r_2]               {};
    \node[]  (r_3)[right of =r_21]               {\textcolor{darkred}{$R_3$}};

    \node[]  (t_0)[below of =time]                {0:$\emptyset$};
    \node[]  (t_1)[below of =t_0]                 {1:$\{a,b\}$};

    \node[]  (s_01)[below of =r_1]                 {$\varphi$};
    \node[]  (s_31)[below of =s_01, pin=below:\tpbf]                 {$a$};
    \node[]  (s_21)[right of =s_31, pin=below:\tpbf]                 {$b$};
    \node[]  (s_02)[below of =r_2]                 {$\varphi$};
    \node[]  (s_32)[below of =s_02, pin=below:\tpbf]                 {$a$};
    \node[]  (s_12)[right of =s_32, pin=below:\textcolor{darkred}{$\fpbf$}]                 {$c$};
    \node[]  (s_03)[below of =r_3]                 {$\varphi$};
    \node[]  (s_33)[below of =s_03, pin=below:\tpbf]                 {$a$};
    \node[]  (s_13)[right of =s_33, pin=below:\textcolor{darkred}{$\fpbf$}]                 {$c$};
    \node[]  (s_23)[right of =s_13, pin=below:\tpbf]                 {$b$};
   
    \path[->,dotted]
    (s_01) edge                node {}  (s_31)
           edge                node {}  (s_21)
    (s_02) edge                node {}  (s_32)
           edge                node {}  (s_12)
    (s_03) edge                node {}  (s_33)
           edge                node {}  (s_13)
           edge                node {}  (s_23);

\end{tikzpicture}
    

\end{example}

%------------------ Example not N(not a or b)--------

\begin{example}[Translation without branching]
    Given the formula $\varphi=\Next(a\wedge \neg b)$. The AFW$_f$ automaton for $\varphi$ is $A_\varphi =(2^{\{a,b,\last\}},S,``\varphi",\delta,\emptyset)$ where $S=\{``\varphi",``a\wedge \neg b",``a",``\neg b", ``b"\}$. We present the transition diagram bellow. We omit the state for $b$ as it is not connected to the system.

    \begin{tikzpicture}[shorten >=1pt,node distance=1.7cm,on grid,auto]
    \tikzstyle{every state}=[fill={rgb:black,1;white,20},font=\small]
    \tikzstyle{every pin edge}=[->,shorten <=1pt]
    \tikzstyle{interp}=[font=\footnotesize]
    \tikzstyle{conn}=[rectangle,draw,font=\small]
    \tikzstyle{bool}=[draw,font=\small]

    \node[state,initial]  (s_0)                 {$\varphi$};
    \node[conn]                    (check_l) [right of=s_0]  {$?$};
    \node[state]                    (s_and) [below of=check_l]  {$a \wedge \neg b$};
    \node[conn]                    (and) [right of=s_and]  {$\wedge$};
    \node[conn]                    (check_a) [below left of=and]  {$?$};
    \node[conn]                    (check_b) [below right of=and]  {$?$};
    \node (false_l) [right of=check_l] {\fpbf};
    \node (true_a) [below left of=check_a] {\tpbf};
    \node (false_a) [below right of=check_a] {\fpbf};
    \node (true_b) [below right of=check_b] {\tpbf};

    \node[state,fill=darkblue!20]                    (s_a) [left of=check_a]  {$a$};
    \node[state,fill=darkblue!20]                    (s_b) [right of=check_b]  {$\neg b$};

  
    \path[->]
    (s_0) edge  (check_l)
    (check_l) edge node[interp] {$\interp{\last }$} (false_l)
              edge node[interp] {$\interp{\neg \last }$} (s_and)
    (s_and) edge (and)
    (and) edge (check_a)          
          edge (check_b)
    (check_a) edge node[interp] {$\interp{a }$} (true_a)
              edge node[interp] {$\interp{\neg a }$} (false_a)          
    (check_b) edge node[interp] {$\interp{\neg b }$} (true_b)
              edge node[interp] {$\interp{b }$} (false_a) 
    (s_a) edge (check_a)
    (s_b) edge (check_b)         
    ;
\end{tikzpicture}

    % The next cases of the transition functions follow directly from the definition.

    % \begin{multicols}{2}
    %     \begin{itemize}
    %         \item $\delta(``a",\interp{ a })=\bm{true}$
    %         \item $\delta(``a",\interp{ \neg a })=\bm{\mathit{false}}$
    %         \item $\delta(``b",\interp{ b })=\bm{true}$
    %         \item $\delta(``b",\interp{ \neg b })=\bm{\mathit{false}}$
    %         \item $\delta(``\neg b",\interp{ b })=\bm{\mathit{false}}$
    %         \item $\delta(``\neg b",\interp{ \neg b })=\bm{true}$
    %         \item $\delta(``\varphi",\interp{ \last })=\bm{\mathit{false}}$
    %         \item $\delta(``\varphi",\interp{ \neg \last })=``a \wedge \neg b"$
    %     \end{itemize}
    % \end{multicols}

    To compute the transition from state $``a \wedge \neg b"$ we must analyze different interpretations and use the trivial base cases previously calculated.

    \begin{itemize}
        \small
        \item $\delta(``a \wedge \neg b",\interp{ \neg b,a })= \delta(``a",\interp{ \neg b,a }) \wedge \delta(``\neg b",\interp{ \neg b,a })\\
        \phantom{\delta(``a \wedge \neg b",\interp{ \neg b,a })} \equiv \bm{true} \wedge \bm{true} \equiv \bm{true}$


        \item $\delta(``a \wedge \neg b",\interp{ b })=
        \delta(``a",\interp{ b }) \wedge \delta(``\neg b",\interp{ b })\\
        \phantom{\delta(``a \wedge \neg b",\interp{ b })} \equiv \delta(``a",\interp{ b }) \wedge \fpbf \equiv \fpbf$
        
        \item $\delta(``a \wedge \neg b",\interp{ \neg a })=
        \delta(``a",\interp{ b }) \wedge \delta(``\neg b",\interp{ \neg a })\\
        \phantom{\delta(``a \wedge \neg b",\interp{ \neg a })} \equiv \fpbf \wedge \delta(``b",\interp{ \neg a })  \equiv \fpbf$      
    \end{itemize}

    These definitions lead to the following alternating automaton, simplifying the transition system.

    \begin{tikzpicture}[shorten >=1pt,node distance=2cm,on grid,auto]
    \tikzstyle{every state}=[fill={rgb:black,1;white,20},font=\small]
    \tikzstyle{every pin edge}=[->,shorten <=1pt]
    \tikzstyle{interp}=[font=\footnotesize]
    \tikzstyle{conn}=[rectangle,draw,font=\small]
    \tikzstyle{bool}=[draw,font=\small]


    \node[state,initial]  (s_0)                 {$\varphi$};
    \node (false_l) [right of=s_0] {\fpbf};
    \node[state] (s_and) [below of=false_l]  {$a \wedge \neg b$};
    \node (true_1) [left of=s_and] {\tpbf};
    \node (false_1) [right of=s_and] {\fpbf};
    \node (false_2) [right of=false_1] {\fpbf};
    
    \path[->]
    (s_0) edge[bend left=10] node[interp] {$\interp{ \last }$} (false_l)
          edge[bend right=10] node[interp] {$\interp{ \neg \last }$} (s_and)
    (s_and) edge[bend right=10] node[interp] {$\interp{ a, \neg b }$} (true_1)
    (s_and) edge[bend left=10] node[interp] {$\interp{ \neg a }$} (false_1)
    (s_and) edge[bend right] node[interp] {$\interp{ b }$} (false_2);
\end{tikzpicture}


    For the trace $\pi=\{a\},\{a,\last\}$, automaton $A_{\varphi}$ has a single run $R_1$ which is accepted.

    \begin{tikzpicture}[shorten >=1.5pt,node distance=1.5cm,on grid,auto]
    % \tikzstyle{every state}=[fill={rgb:black,1;white,20}]
    \tikzstyle{every pin edge}=[->,dotted,shorten <=1pt]
    % \tikzstyle{conn}=[rectangle,draw]
    % \tikzstyle{bool}=[rectangle,draw]
    \node[]  (time)                 {\textcolor{darkblue}{TIME}};
    \node[]  (r_1)[right of =time]               {\textcolor{darkblue}{$R_1$}};

    \node[]  (t_0)[below of =time]                {0:$\{a\}$};
    \node[]  (t_1)[below of =t_0]                 {1:$\{a, \last\}$};

    \node[]  (s_phi)[right of =t_0]                 {$\varphi$};
    \node[]  (s_and)[below of =s_01, pin=below:\tpbf]                 {$a \wedge \neg b$};
   
    \path[->,dotted]
    (s_phi) edge                node {}  (s_and);

\end{tikzpicture}
    

\end{example}

%------------------ Example always A --------
\begin{example}[Always operator]
    Given the formula $\varphi= \bot \release a \defeq \alwaysF a$, the corresponding AFW$_f$ automaton for $\varphi$ is $A_\varphi =(2^{\{a,\last\}},S,\s{\varphi},\delta,\emptyset)$ where $S=\{\s{\varphi},\s{a},\s{\bot}\}$. 
    
    % The next transition functions follow directly from the definition.

    % \begin{multicols}{2}
    %     \begin{itemize}
    %         \item $\delta(``a",\interp{ a })=\bm{true}$
    %         \item $\delta(``a",\interp{ \neg a })=\bm{\mathit{false}}$
    %         \item $\delta(``\bot",\Pi)=\bm{\mathit{false}}$
    %     \end{itemize}
    % \end{multicols}

    To compute the transition from state $``\bot \release a"$, we must analyze different interpretations and use the trivial base cases previously calculated.

    \begin{itemize}
        \small
    
        \item $\delta(``\bot \release a",\interp{ \last,a }) = \delta(``a",\interp{ \last,a }) \equiv \bm{true}$
        
        \item $\delta(``\bot \release a",\interp{ \last, \neg a }) = \delta(``a",\interp{ \last, \neg a }) \equiv \fpbf$
    
        \item $\delta(``\bot \release a",\interp{ \neg \last, \neg a }) = \delta(``a",\interp{ \neg \last, \neg a }) \wedge \\ 
        \phantom{\delta(``\bot \release a",\interp{ \neg \last, \neg a })=}
        (\delta(``\bot",\interp{ \neg \last, \neg a })  \vee ``\bot \release a")  \\
        \phantom{\delta(``\bot \release a",\interp{ \neg \last, \neg a })} \equiv 
        \fpbf \wedge (\fpbf  \vee ``\bot \release a")  \equiv $


        \item $\delta(``\bot \release a",\interp{ \neg \last, a }) = \delta(``a",\interp{ \neg \last, a }) \wedge (\delta(``\bot",\interp{ \neg \last, a })  \vee ``\bot \release a")  \\
        \phantom{\delta(``\bot \release a",\interp{ \neg \last, a })} \equiv 
        \bm{true} \wedge (\fpbf  \vee ``\bot \release a") \equiv ``\bot \release a"$
    \end{itemize}

    These definitions lead to the following alternating automaton.

    \begin{tikzpicture}[shorten >=1pt,node distance=2cm,on grid,auto]
    \tikzstyle{every state}=[fill={rgb:black,1;white,20},font=\small]
    \tikzstyle{every pin edge}=[->,shorten <=1pt]
    \tikzstyle{interp}=[font=\footnotesize]
    \tikzstyle{conn}=[rectangle,draw,font=\small]
    \tikzstyle{bool}=[draw,font=\small]
    \node[state,initial]  (s_0)                 {$\bot \release a$};
    \node (false_l) [right of=s_0] {\fpbf};
    \node (true_1) [right of=false_l] {\tpbf};
    
    \path[->]
    (s_0) edge[bend left=10] node[interp] {$\interp{ \neg a }$} (false_l)
          edge[bend right=30] node[interp] {$\interp{ \last, a }$} (true_1)
          edge[loop above] node[interp] {$\interp{ \neg \last, a }$} (s_0);
\end{tikzpicture}

\end{example}




%%%%%%%%%%%%%%%%% LDLf %%%%%%%%%%%%%%%%%%%%%%%%%
\subsubsection{$\LDLf$ into alternating automata}

We now show how we can associate a dynamic formula $\varphi$ in \nnf with an $\AFW$ $A_\varphi$, where the state-size of $A_\varphi$ is linear in the size of $\varphi$, such that $L(A_\varphi)$ is exactly the set of traces that satisfy $\varphi$. For this translation, we first extend the notion of the closure of a formula by using the Fisher-Ladner closure\footnote{We omit the case for $\deventually{p}\gamma$ with $p$ propositional as this is consider syntactic sugar in our grammar and can be expressed as $\deventually{p?;\top}\gamma$. The box operator is defined analogously to the diamond operator.} \cite{fislad79a}:

\begin{definition}[Closure for $\LDLf$]
    The closure $cl(\varphi)$ of a dynamic formula $\varphi$ is the smallest set of dynamic formulas inductively defined as follows:
    
    \begin{itemize}
        \item $\varphi \in cl(\varphi)$
        \item $\neg \psi \in cl(\varphi)$ if $\psi \in cl(\varphi)$ and  $\psi$ is not of the form $\neg \psi'$
        \item if $\deventually{\rho}\psi \in cl(\varphi)$ then $\psi\in cl(\varphi)$ 
        \item if $\deventually{\gamma?}\psi \in cl(\varphi)$ then $\gamma \in cl(\varphi)$
        \item if $\deventually{\rho_1;\rho_2}\psi \in cl(\varphi)$ then $\deventually{\rho_1}\deventually{\rho_2}\psi \in cl(\varphi)$
        \item if $\deventually{\rho_1+\rho_2}\psi \in cl(\varphi)$ then $\deventually{\rho_1}\psi \in cl(\varphi)$ and $\deventually{\rho_2}\psi \in cl(\varphi)$
        \item if $\deventually{\rho^*}\psi \in cl(\varphi)$ then $\deventually{\rho}\deventually{\rho*}\psi \in cl(\varphi)$
    \end{itemize}
\end{definition}


Based on the previous definition, we consider as states all quoted formulas $``\psi"$ where $\psi \in cl(\varphi)$. Like for $\LTLf$, we include the special proposition $\last$ and use it along with the set of atoms $\A$ to construct the alphabet $\Sigma=2^{\A \cup \{\last\}}$.

\begin{definition}[$\LDLf$ to $\AFW$]
    The $\AFW$ for a dynamic formula $\varphi$ in \nnf is defined as:
    $A_\varphi =(2^{\A \cup \{\last\}},\{``\psi" \mid \psi \in cl(\varphi) \},``\varphi",\delta,\emptyset)$, where the transition function $\delta = S\times \Sigma \to B^+(S)$ from a quoted formula and interpretation $\Pi\in\Sigma$ is defined as follows: 


%Cardinality linear in the size of phi
  \begin{multicols}{2}
    \small
    \begin{itemize}
 
      
    \item $\delta(\s{\top},\Pi)=\tpbf$
            
      \item $\delta(\s{\deventually{\top}\varphi},\Pi)=\begin{cases}
        \s{\varphi} & \text{if} \;\last\not\in \Pi \\
        \fpbf & \text{if} \;\last\in \Pi \\
      \end{cases}
      $
      

      
      \item $\delta(\s{\bot},\Pi)=\fpbf$
      
      \item $\delta(\s{\dalways{\top}\varphi},\Pi)=\begin{cases}
        \varphi & \text{if} \;\last\not\in \Pi \\
        \tpbf & \text{if} \;\last\in \Pi \\
      \end{cases}
      $

    \end{itemize}
  \end{multicols}
\begin{itemize}

          
  \item $\delta(\s{\deventually{\psi ?}\varphi},\Pi)=
  \delta(\s{\psi},\Pi) \; \wedge \delta(\s{\varphi},\Pi)
  $
  
  \item $\delta(\s{\deventually{\rho_1 + \rho_2}\varphi},\Pi)=
  \delta(\s{\deventually{\rho_1}\varphi},\Pi) \vee \; \delta(\s{\deventually{\rho_2}\varphi},\Pi)
  $
  
  \item $\delta(\s{\deventually{\rho_1 ; \rho_2}\varphi},\Pi)=
    \delta(\s{\deventually{\rho_1}\deventually{\rho_2}\varphi},\Pi)
    $

  \item $\delta(\s{\deventually{\rho^{\ast}}\varphi},\Pi)= 
  \delta(\s{\varphi},\Pi)\; \vee \delta(\s{\deventually{\rho}\deventually{\rho^{\ast}}\varphi},\Pi)
  $
  
  
  \item $\delta(\s{\deventually{(\psi ?)^{\ast}}\varphi},\Pi)=
  \delta(\s{\varphi},\Pi)
  $
      

  \item $\delta(\s{\dalways{\psi ?}\varphi},\Pi)=
  \delta(\s{\neg \psi},\Pi)\; \vee \delta(\s{\varphi},\Pi)
  $
  
  \item $\delta(\s{\dalways{\rho_1 + \rho_2}\varphi},\Pi)=
  \delta(\s{\dalways{\rho_1}\varphi},\Pi) \wedge \; \delta(\s{\dalways{\rho_2}\varphi},\Pi)
  $

              
      
  \item $\delta(\s{\dalways{\rho_1 ; \rho_2}\varphi},\Pi)=
  \delta(\s{\dalways{\rho_1}\dalways{\rho_2}\varphi},\Pi)
  $

        
  \item $\delta(\s{\dalways{\rho^{\ast}}\varphi},\Pi)=
  \delta(\s{\varphi},\Pi)\; \wedge \delta(\s{\dalways{\rho}\dalways{\rho^{\ast}}\varphi},\Pi)
  $


      
  \item $\delta(\s{\dalways{(\psi ?)^{\ast}}\varphi},\Pi)=
  \delta(\s{\varphi},\Pi)
  $

\end{itemize}
\end{definition}

The definition for the transition brought to light the need of an additional adjustment to the syntax for dynamic formulas and to the closure. 
Such adaptation was needed since the definition provided by De Giacomo and Vardi in \cite{giavar15a}: $\delta(\s{\dalways{\psi ?}\varphi},\Pi) = \delta(\s{\nnf(\neg \psi)},\Pi) \vee \delta(\s{\varphi},\Pi)$ does not take into account that the formulas generated by $\nnf(\neg \psi)$ should also be part of the closure.  

\begin{example}[Problematic definiton for test construct]
    Given the formula $\varphi = \dalways{(\deventually{\top}a)?}b$ where the test construct is applied to a non-propositional formula, we would calculate the set of states using the closure as $S=\{\s{\varphi}, \s{\neg \varphi} \s{b}, \s{\neg b}, \s{\deventually{\top}a}, \s{\neg \deventually{\top}a}, \s{\top}, \s{a}, \s{\neg a}, \s{\bot}\}$. Given the original definition by De Giacomo and Vardi we would have the following transition: 
    
    $\delta(\s{\dalways{(\deventually{\top}a)?}b},\Pi) = \delta(\s{\nnf(\neg \deventually{\top}a)},\Pi) \vee \delta(\s{b},\Pi) \\\phantom{\delta(\s{\dalways{(\deventually{\top}a)?}b},\Pi)\;} \; \equiv \delta(\s{\dalways{\top}\neg a},\Pi) \vee \delta(\s{b},\Pi)$ .

    In this transition function, we are referring to state $\s{\dalways{\top}\neg a}$ which was not included as part of the closure. Even tough the equivalent formula $\neg \deventually{\top}a$ was included, the reduction into $\nnf$ as well as all subformulas that this process may convey are not considered as part of the states.
\end{example}

We tackle the problem by restricting the test construct only to propositions and by directly applying a negation instead of \nnf in the translation, as it is shown on the definition above.
With this restriction, we only need to include the negation for propositions as part of the closure, resulting in exclusively formulas in \nnf. 
We argue that this remodelling does not impose limitations for formulas for which the subformula inside the test construct can be redefined as a new proposition $p_{\varphi}$ where $p_\varphi \leftrightarrow \varphi$. 
For instance, rewriting the formula $\dalways{(\neg p) ?} q$ as $\dalways{l_{\neg p} ?} q$ and including the rule $l_{\neg p} \leftarrow not \;p$ \footnote{For this formula the other side of the implication is directly given by the closed world assumption since $l_{\neg p}$ is a new proposition and therefore can only be inferred by this rule.}.
However, when the test is applied to a more complex formula, it is not longer trivial to rewrite it without using dynamic operators. 
Such a statement is based on the fact that we restricted dynamic formulas to integrity constraints, thus, we can't use them to generate new predicates. 
Therefore, a complete reanalysis of the transition is required to account for the full expressiveness of this logic.
Such analysis was outside of the scope for this thesis, leaving us with the decision to move on to the implementation using the restricted logic.


% This remodelling does not impose limitations, since any formula $\varphi$ required in the test construct $\varphi ?$ can redefined as a proposition $p_{\varphi}$ using an additional rule $p_\varphi \leftarrow \varphi$ as part of the encoding.

\newpage
%------------------ Example always A --------
\begin{example}[Always operator]
    Given the formula $\varphi= \dalways{\top ^*}a \equiv \alwaysF a$, the AFW$_f$ automaton for $\varphi$ is $A_\varphi =(2^{\{a,\last\}},S,\s{\varphi},\delta,\emptyset)$ where $S=\{\s{\varphi},\s{\dalways{\top}\dalways{\top ^*}a},``a"\}$. The base cases required from the transition $\delta$ are the following:

    \begin{multicols}{2}
        \small
        \begin{itemize}
            \item $\delta(``a",\interp{ a })=\bm{true}$
            \item $\delta(``a",\interp{ \neg a })=\bm{\mathit{false}}$
        \end{itemize}
    \end{multicols}
    \begin{itemize}
        \small
        \item $\delta(\s{\dalways{\top}\dalways{\top ^*}a},\interp{ \neg \last})=\tpbf$
        \item $\delta(\s{\dalways{\top}\dalways{\top ^*}a},\interp{\last})=\s{\dalways{\top ^*}a}$
    \end{itemize}

    To compute the transition of state $\s{\dalways{\top ^*}a}$ we must analyze different interpretations and use the definitions calculated above.

    \begin{itemize}
        
        \small
        \item $\delta(\s{\dalways{\top ^*}a},\interp{\neg a}) = \delta(\s{a},\interp{\neg a}) \wedge \delta(\s{\dalways{\top}\dalways{\top ^*}a},\interp{\neg a}) \\\phantom{\delta(\s{\dalways{\top ^*}a},\interp{\neg a})} \equiv  \fpbf \wedge \delta(\s{\dalways{\top}\dalways{\top ^*}a},\interp{\neg a}) \equiv \fpbf$ 
        
        \item $\delta(\s{\dalways{\top ^*}a},\interp{a, \last}) = \delta(\s{a},\interp{a, \last}) \wedge \delta(\s{\dalways{\top}\dalways{\top ^*}a},\interp{a, \last}) \\\phantom{\delta(\s{\dalways{\top ^*}a},\interp{a, \last})} \equiv  \tpbf \wedge \tpbf \equiv \tpbf$ 
        
        \item $\delta(\s{\dalways{\top ^*}a},\interp{a, \neg \last}) = \delta(\s{a},\interp{a, \neg \last}) \wedge \delta(\s{\dalways{\top}\dalways{\top ^*}a},\interp{a, \neg \last}) \\\phantom{\delta(\s{\dalways{\top ^*}a},\interp{a, \neg \last})} \equiv  \tpbf \wedge \s{\dalways{\top ^*}a} \equiv \s{\dalways{\top ^*}a}$ 
        
        
    \end{itemize}

    These definitions lead to the following alternating automaton, identical to the one shown in Example 9.
    
\begin{tikzpicture}[shorten >=1pt,node distance=2cm,on grid,auto]
        \tikzstyle{every state}=[fill={rgb:black,1;white,20},font=\small]
        \tikzstyle{every pin edge}=[->,shorten <=1pt]
        \tikzstyle{interp}=[font=\footnotesize]
        \tikzstyle{conn}=[rectangle,draw,font=\small]
        \tikzstyle{bool}=[draw,font=\small]


        \node[state,initial]  (s_0)                 {$\dalways{\top ^*}a$};
        \node (false_l) [right of=s_0] {\fpbf};
        \node (true_1) [right of=false_l] {\tpbf};
        
        \path[->]
        (s_0) edge[bend left=10] node[interp] {$\interp{ \neg a }$} (false_l)
              edge[bend right=30] node[interp] {$\interp{ \last, a }$} (true_1)
              edge[loop above] node[interp] {$\interp{ \neg \last, a }$} (s_0);
\end{tikzpicture}

\end{example}



%------------------ Example Final --------
\begin{example}[Final operator]
    Given the formula $\varphi= \dalways{\top}\bot \equiv \finally$, the AFW$_f$ automaton for $\varphi$ is $A_\varphi =(2^{\{\last\}},S,\s{\varphi},\delta,\emptyset)$ where $S=\{\s{\varphi},\s{\bot}\}$, where the transition $\delta$ is defined as:

    \begin{itemize}
        \item $\delta(\s{\dalways{\top}\bot},\interp{\last}) = \tpbf$ 
        \item $\delta(\s{\dalways{\top}\bot},\interp{\neg \last}) = \delta(\s{\bot}) = \fpbf$         
    \end{itemize}

    These definitions lead to the following alternating automaton, showing the expected behavior of hitting the true transition only when \last is in the interpretation.

    \begin{tikzpicture}[shorten >=1pt,node distance=2cm,on grid,auto]
    \tikzstyle{every state}=[fill={rgb:black,1;white,20},font=\small]
    \tikzstyle{every pin edge}=[->,shorten <=1pt]
    \tikzstyle{interp}=[font=\footnotesize]
    \tikzstyle{conn}=[rectangle,draw,font=\small]
    \tikzstyle{bool}=[draw,font=\small]

    \node[state,initial]  (s_0)                 {$\dalways{\top}\bot$};
    \node (false_l) [right of=s_0] {\fpbf};
    \node (true_1) [right of=false_l] {\tpbf};
    
    \path[->]
    (s_0) edge[bend left=10] node[interp] {$\interp{\neg last}$} (false_l)
          edge[bend right=30] node[interp] {$\interp{last}$} (true_1);
\end{tikzpicture}

\end{example}


%------------------ Example diamond sequence choice --------
\begin{example}[Diamond, sequence and choice operators]
    Given the formula $\varphi= \deventually{((a? + b?);\top);b?}\finally$, the AFW$_f$ automaton for $\varphi$ is $A_\varphi =(2^{\{a,b,\last\}},S,"\varphi",\delta,\emptyset)$ where the set of states $S$ consists of the following formulas after before quotes.
    \begin{multicols}{3}
        \small
        \begin{itemize}
        
            \item$\varphi$
            \item$\deventually{(a? + b?);\top}\deventually{b?}\finally$
            \item$\deventually{(a? + b?)}\deventually{\top}\deventually{b?}\finally$
            \item$\deventually{a?}\deventually{\top}\deventually{b?}\finally$
            \item$\deventually{b?}\deventually{\top}\deventually{b?}\finally$
            \item$\deventually{\top}\deventually{b?}\finally$
            \item$\deventually{b?}\finally$ 
            \item$\finally$ 
            \item$\deventually{\top;b?}\finally$
            \item$a$
            \item$b$
            \item$\bot$
        \end{itemize}
    \end{multicols}
    
    The transition $\delta$ is calculated below, omitting base cases and using the transitions from the last example.
    \small
    \begin{itemize}

        \item $\delta(\s{\deventually{b?}\finally},\interp{\neg \last})=
        \delta(\s{b},\interp{\neg \last}) \; \wedge \;
         \delta(\s{\finally},\interp{\neg \last})\\
        \phantom{\delta(\s{\deventually{b?}\finally},\interp{\neg \last})}\equiv
        \delta(\s{b},\interp{\neg \last})\;\wedge\;\fpbf \equiv \fpbf
        $

        \item $\delta(\s{\deventually{b?}\finally},\interp{\last,b})=
        \delta(\s{b},\interp{\last,b}) \; \wedge \;
         \delta(\s{\finally},\interp{\last,b})\\
        \phantom{\delta(\s{\deventually{b?}\finally},\interp{\last,b})}\equiv
        \tpbf\;\wedge\;\tpbf \equiv \tpbf
        $

        \item $\delta(\s{\deventually{b?}\finally},\interp{\last,\neg b})=
        \delta(\s{b},\interp{\last,\neg b}) \; \wedge \;
         \delta(\s{\finally},\interp{\last,\neg b})\\
        \phantom{\delta(\s{\deventually{b?}\finally},\interp{\last,\neg b})}\equiv
        \fpbf\;\wedge\;\tpbf \equiv \fpbf
        $
        

        \item $\delta(\s{\deventually{b?}\deventually{\top}\deventually{b?}\finally},\interp{\last})=
        \delta(\s{b},\interp{\last}) \; \wedge \;
         \delta(\s{\deventually{\top}\deventually{b?}\finally},\interp{\last})\\
        \phantom{\delta(\s{\deventually{b?}\deventually{\top}\deventually{b?}\finally},\interp{\last})}\equiv
        \delta(\s{b},\interp{\last})\;\wedge\;\fpbf \equiv \fpbf
        $


        \item $\delta(\s{\deventually{b?}\deventually{\top}\deventually{b?}\finally},\interp{\neg \last,b})=
        \delta(\s{b},\interp{\neg \last,b}) \; \wedge\;
         \delta(\s{\deventually{\top}\deventually{b?}\finally},\interp{\neg \last,b})\\
        \phantom{\delta(\s{\deventually{b?}\deventually{\top}\deventually{b?}\finally},\interp{\neg \last,b})}\equiv
        \tpbf \wedge \s{\deventually{b?}\finally} \equiv \s{\deventually{b?}\finally}\\
        $

        \item $\delta(\s{\deventually{b?}\deventually{\top}\deventually{b?}\finally},\interp{\neg \last, \neg b})=
        \delta(\s{b},\interp{\neg \last, \neg b}) \; \wedge\;
         \delta(\s{\deventually{\top}\deventually{b?}\finally},\interp{\neg \last, \neg b})\\
        \phantom{\delta(\s{\deventually{b?}\deventually{\top}\deventually{b?}\finally},\interp{\neg \last, \neg b})}\equiv
        \fpbf \wedge \s{\deventually{b?}\finally} \equiv \fpbf\\
        $

        \item $\delta(\s{\deventually{a?}\deventually{\top}\deventually{b?}\finally},\interp{\last})=
        \delta(\s{a},\interp{\last}) \; \wedge \;
         \delta(\s{\deventually{\top}\deventually{b?}\finally},\interp{\last})\\
        \phantom{\delta(\s{\deventually{a?}\deventually{\top}\deventually{b?}\finally},\interp{\last})}\equiv
        \delta(\s{a},\interp{\last})\;\wedge\;\fpbf \equiv \fpbf
        $


        \item $\delta(\s{\deventually{a?}\deventually{\top}\deventually{b?}\finally},\interp{\neg \last,a})=
        \delta(\s{a},\interp{\neg \last,a}) \; \wedge\;
         \delta(\s{\deventually{\top}\deventually{b?}\finally},\interp{\neg \last,a})\\
        \phantom{\delta(\s{\deventually{b?}\deventually{\top}\deventually{b?}\finally},\interp{\neg \last,a})}\equiv
        \tpbf \wedge \s{\deventually{b?}\finally} \equiv \s{\deventually{b?}\finally}\\
        $

        \item $\delta(\s{\deventually{a?}\deventually{\top}\deventually{b?}\finally},\interp{\neg \last, \neg a})=
        \delta(\s{a},\interp{\neg \last, \neg a}) \; \wedge\;
         \delta(\s{\deventually{\top}\deventually{b?}\finally},\interp{\neg \last, \neg a})\\
        \phantom{\delta(\s{\deventually{a?}\deventually{\top}\deventually{b?}\finally},\interp{\neg \last, \neg a})}\equiv
        \fpbf \wedge \s{\deventually{b?}\finally} \equiv \fpbf\\
        $

        \item $\delta(\s{\deventually{a?+b?}\deventually{\top}\deventually{b?}\finally},\interp{\last})=
        \delta(\s{\deventually{a?}\deventually{\top}\deventually{b?}\finally},\interp{\last}) \; \vee\\
        \phantom{\delta(\s{\deventually{a?+b?}\deventually{\top}\deventually{b?}\finally},\interp{\last})=}
        \delta(\s{\deventually{b?}\deventually{\top}\deventually{b?}\finally},\interp{\last})\\
        \phantom{\delta(\s{\deventually{a?+b?}\deventually{\top}\deventually{b?}\finally},\interp{\last})}\equiv
        \fpbf \wedge \fpbf \equiv \fpbf\\
        $

        \item $\delta(\s{\deventually{a?+b?}\deventually{\top}\deventually{b?}\finally},\interp{\neg \last,a,b})=
        \delta(\s{\deventually{a?}\deventually{\top}\deventually{b?}\finally},\interp{\neg \last,a,b}) \; \vee\\
        \phantom{\delta(\s{\deventually{a?+b?}\deventually{\top}\deventually{b?}\finally},\interp{\neg \last,a,b})=}
        \delta(\s{\deventually{b?}\deventually{\top}\deventually{b?}\finally},\interp{\neg \last,a,b})\\
        \phantom{\delta(\s{\deventually{a?+b?}\deventually{\top}\deventually{b?}\finally},\interp{\neg \last,a,b})}\equiv
        \s{\deventually{b?}\finally} \vee \s{\deventually{b?}\finally} \equiv \s{\deventually{b?}\finally}\\
        $

        \item $\delta(\s{\deventually{a?+b?}\deventually{\top}\deventually{b?}\finally},\interp{\neg \last,\neg a,b})=
        \delta(\s{\deventually{a?}\deventually{\top}\deventually{b?}\finally},\interp{\neg \last,\neg a,b}) \; \vee\\
        \phantom{\delta(\s{\deventually{a?+b?}\deventually{\top}\deventually{b?}\finally},\interp{\neg \last,\neg a,b})=}
        \delta(\s{\deventually{b?}\deventually{\top}\deventually{b?}\finally},\interp{\neg \last,\neg a,b})\\
        \phantom{\delta(\s{\deventually{a?+b?}\deventually{\top}\deventually{b?}\finally},\interp{\neg \last,\neg a,b})}\equiv
        \fpbf \vee \s{\deventually{b?}\finally} \equiv \s{\deventually{b?}\finally}\\
        $

        \item $\delta(\s{\deventually{a?+b?}\deventually{\top}\deventually{b?}\finally},\interp{\neg \last,\neg a,\neg b})=
        \delta(\s{\deventually{a?}\deventually{\top}\deventually{b?}\finally},\interp{\neg \last,\neg a,\neg b}) \; \vee\\
        \phantom{\delta(\s{\deventually{a?+b?}\deventually{\top}\deventually{b?}\finally},\interp{\neg \last,\neg a,\neg b})=}
        \delta(\s{\deventually{b?}\deventually{\top}\deventually{b?}\finally},\interp{\neg \last,\neg a,\neg b})\\
        \phantom{\delta(\s{\deventually{a?+b?}\deventually{\top}\deventually{b?}\finally},\interp{\neg \last,\neg a,\neg b})}\equiv
        \fpbf \vee \fpbf \equiv \fpbf\\
        $

        \item $\delta(\varphi,\Pi) = \delta(\s{\deventually{a?+b?}\deventually{\top}\deventually{b?}\finally},
        \Pi)
        $

    \end{itemize}

    These definitions lead to the following alternating automaton.

    \begin{tikzpicture}[shorten >=1pt,node distance=2cm,on grid,auto]
    \tikzstyle{every state}=[fill={rgb:black,1;white,20},font=\small]
    \tikzstyle{every pin edge}=[->,shorten <=1pt]
    \tikzstyle{interp}=[font=\footnotesize]
    \tikzstyle{conn}=[rectangle,draw,font=\small]
    \tikzstyle{bool}=[draw,font=\small]
    \node[state,initial]  (s_0)                 {$\varphi$};
    \node (s_0_e) [right of=s_0] {};
    \node (false_0) [below of=s_0] {\fpbf};
    \node[state]  (s_f) [right of =s_0_e]                {$\deventually{b?}\finally$};
    \node (s_f_e) [right of=s_f] {};
    \node (true_1) [below of=s_f] {\tpbf};
    \node (false_1) [right of=s_f_e] {\fpbf};
    
    \path[->]
    (s_0) edge[bend right=20] node[interp] {$\interp{\last},\interp{\neg \last, \neg a, \neg b}$} (false_0)
            edge[bend left=30] node[interp] {$\interp{\neg \last, a},\interp{\neg \last, b}$} (s_f)

    (s_f) edge[bend right=30] node[interp] {$\;\;\interp{\last, b}$} (true_1)
    edge[bend left=10] node[interp] {$\interp{\neg \last},\interp{\neg b}$} (false_1);
\end{tikzpicture}

    Traces $\pi_1=\{a\},\{b,\last\}$, $\pi_2=\{b\},\{b,\last\}$, and $\pi_3=\{b,a\},\{b,\last\}$ are all accepted by the automaton generating the accepted run $R_1$. On the other hand, traces such as $\pi_4=\{a,\last\}$ and $\pi_5=\{b\},\{b\},\{a,\last\}$ are not accepted yielding the runs $R_2$ and $R_3$ respectively.

    \begin{tikzpicture}[shorten >=1.5pt,node distance=1.5cm,on grid,auto]
    % \tikzstyle{every state}=[fill={rgb:black,1;white,20}]
    \tikzstyle{every pin edge}=[->,dotted,shorten <=1pt]
    % \tikzstyle{conn}=[rectangle,draw]
    % \tikzstyle{bool}=[rectangle,draw]
    \node[]  (time)                 {\textcolor{darkblue}{TIME}};
    \node[]  (r_1)[right of =time]               {\textcolor{darkblue}{$R_1$}};
    \node[]  (r_1_e)[right of =r_1]               {};

    \node[]  (t_0)[below of =time]                {0};
    \node[]  (t_1)[below of =t_0]                 {1};

    \node[]  (s_phi)[right of =t_0]                 {$\varphi$};
    \node[]  (s_and)[below of =s_phi, pin=below:\tpbf]                 {$\deventually{b?}\finally$};
   
    \path[->,dotted]
    (s_phi) edge                node {}  (s_and);


    \node[]  (time_2)[right of =r_1_e]                 {\textcolor{darkblue}{TIME}};
    \node[]  (r_2)[right of =time_2]               {\textcolor{darkred}{$R_2$}};
    \node[]  (r_2_e)[right of =r_2]               {};

    \node[]  (t_0_1)[below of =time_2]                {0:$\{a,\last\}$};

    \node[]  (s_phi_2)[right of =t_0_1, pin=below:\fpbf]                 {$\varphi$};
   

    \node[]  (time_3)[right of =r_2_e]                 {\textcolor{darkblue}{TIME}};
    \node[]  (r_1)[right of =time_3]               {\textcolor{darkred}{$R_3$}};

    \node[]  (t_0_3)[below of =time_3]                {0:$\{b\}$};
    \node[]  (t_1_3)[below of =t_0_3]                 {1:$\{b\}$};

    \node[]  (s_phi_3)[right of =t_0_3]                 {$\varphi$};
    \node[]  (s_and_3)[below of =s_phi_3, pin=below:\fpbf]                 {$\deventually{b?}\finally$};
   
    \path[->,dotted]
    (s_phi_3) edge                node {}  (s_and_3);

\end{tikzpicture}

\end{example}