\subsection{Equilibrium logic and stable model semantics  }

\subsubsection{Here-and-There Logic}

The logic of here-and-there (HT) first introduced by \cite{heyting30a} is a propositional monotonic logic that can be used for representing and analyzing logic programs under stable model semantics \cite{pearce96a,ferlif05a}. This section introduces the basics of this logic as well as its connection with Answer Set Programming (ASP).


Given a set of atoms $\A$, a formula $\varphi$ in the logic of HT is constructed with the grammar:
\begin{align*}
    \varphi ::= &\; a \mid \bot \mid
        \varphi \wedge \varphi \mid 
        \varphi \vee \varphi \mid
        \varphi \to \varphi 
\end{align*}
where $a\in \A$ is an atom. As usual, we define a \emph{theory} as a set of formulas, and consider the abbreviations $\neg \varphi \defeq \varphi \to \bot$ and $\top \defeq \bot \to \bot$. An interpretation over $\A$ is a pair $\handt$ where $H\subseteq T \subseteq \A$. We refer to the set $H$ as the ``here'' world and $T$ as the ``there'' world. Intuitively, we can think that atoms in $H$ are \emph{true} while atoms in $T$ are \emph{potentially true}. Based on this observation HT can be used as a tree-valued logic where atoms in $A$ are \emph{true}, those in $\A \backslash T$ are \emph{false} and atoms in $T \backslash H$ are \emph{unknown}. 

\begin{definition}[Semantics]
    An interpretation $\handt$ satisfies a formula $\varphi$, written $\handt \models \varphi$, according to the following definition:
    \begin{itemize}
        \item $\handt \not \models \bot$
        \item $\handt \models a$ iff $a\in H$
        \item $\handt \models \varphi_1 \wedge \varphi_2$ iff $\handt \models \varphi_1$ and  $\handt \models \varphi_2$
        \item $\handt \models \varphi_1 \vee \varphi_2$ iff $\handt \models \varphi_1$ or  $\handt \models \varphi_2$
        \item $\handt \models \varphi_1 \to \varphi_2$ iff $\langle w,T \rangle \not \models \varphi_1$ or  $\langle w,T \rangle \models \varphi_2\;$ for all 
        $w \in \{H,T\}$
    \end{itemize}
\end{definition}

Let us recall some characteristic properties of HT.
\begin{lemma}[Persistence HT \cite{pearce96a}]
    For HT-interpretations $\handt$ and $\tandt$ and formula $\varphi$ over $\A$, $\handt \models \varphi$ implies $\tandt \models \varphi$.
\end{lemma}

\begin{lemma}[Negation \cite{heyting30a}]
    For HT-interpretations $\handt$ and $\tandt$ and formula $\varphi$ over $\A$, $\handt \models \varphi \to \bot$ iff $\tandt \not \models \varphi$.
\end{lemma}

\subsubsection{Equilibrium Logic}

An interpretation $\handt$ is called \emph{total} when $H = T$. 
This type of interpretations enforce classical entitlement, meaning that $\langle T,T \rangle \models \varphi$ can be treated as in classical logic where $T\models \varphi$. 
\begin{definition}[Equilibrium model]
    A total interpretation $\langle T,T\rangle$ is an \emph{equilibrium model} of a theory $\Gamma$ if $\langle T,T\rangle$ is a model of $\Gamma$ and there is no other model $\langle H,T\rangle$ of $\Gamma$ with $H \subset T$.
\end{definition}

The resulting logic is called Equilibrium Logic (EL) \cite{pearce06a}. Such a formalism can be used to represent stable models, so that $T$ is a \emph{stable model} of a theory $\Gamma$ whenever $\langle T,T\rangle$ is an \emph{equilibrium model} of $\Gamma$. 
This characterization of stable models is equivalent to the original one of Gelfond and Lifschitz \cite{gellif88b}. The declarative paradigm of ASP is based on stable models semantics, and thus can be described using EL. 

These formalisms provide a nonmonotonic form of inference that allow us to consider negation as failure, used in the construction of general logic programs \cite{gerosc91a}. 
Negation as failure \cite{clark78a}, written as $\mathit{not}\;a$ , also known as default negation or weak negation, is used to state that $a$ could not be proven. 
% This pseudo-operator can not be directly interpreted, but it needs to be ruled out by a particular syntactic transformation \cite{cabalar01a}. 

\subsubsection{General logic programs}

A general logic program is defined as a set of rules of the form:
\begin{align*}
    \underbrace{a_1 \vee \cdots \vee a_k}_{\mathit{head}} \leftarrow \underbrace{b_{1} \wedge \cdots \wedge b_{m} \wedge \mathit{not}\; c_{1} \wedge \cdots \wedge \mathit{not}\; c_{n}}_{\mathit{body}} 
\end{align*}

\begin{flushleft}
where ${a_i}_{i=1}^k$, ${b_i}_{i=1}^m$ and ${c_i}_{i=1}^n$ are atoms. In practice, the left arrow $\leftarrow$ is represented by the symbol $\texttt{:-}$ and all rules are delimited by a dot. When the body of the rule is empty, meaning $m=0$ and $n=0$, we call it a \emph{fact}. Rules with $k=1$ are called \emph{normal} and with $k=0$ are named \emph{integrity constraints}. 
Integrity constraints have a special behavior under the HT semantics. 
Lets consider the integrity constraint $\leftarrow \mathit{body}$ with $\mathit{body}=l_1 \wedge \cdots \wedge l_m$ where ${l_i}_{i=1}^m$ are \emph{literals} constructed by either atoms or negated atoms. We can rewrite the constraint as $\mathit{body} \to \bot$ and analyze its semantics in HT employing the negation property from Lemma 2. Since $\mathit{body}$ is only checked in $\tandt$, this allows us to conclude that the body of an integrity constraint can be treated as in classical logic. 
\end{flushleft}

\subsubsection{Temporal Here-and-There}


The equilibrium models of $\HT$ can also be extended to deal with dynamic scenarios. Work in this area includes (Linear) Temporal Here-and-There (\THT) \cite{agcadipevi13a} and (Linear) Dynamic logic of Here-and-There (\DHT) \cite{bocadisc18a} as well as their non-monotonic counterpart for temporal stable models Temporal Equilibrium Logic (\TEL) \cite{agcadipevi13a} and Dynamic Equilibrium Logic (\DEL) \cite{cadisc19a}. These logics build upon the logic of HT together with $\LTL$ \cite{pnueli77a} and $\LDL$\cite{giavar13a}, respectively. The idea behind these temporal formalisms is to capture time as sequences of HT-interpretations, called HT-traces \thandt, where their satisfisability is based on the HT-trace for a specific time point $i$, written as $\thandt, i\models \varphi$. Correspondingly, they also fulfill the properties of persistence and negation, described in this case for HT-traces.

While these extensions are able to handle temporal and dynamic formulas with the non-monotonicity from the stable model semantics, we can go back to their classical interpretation by limiting their appearance in logic programs to the body of integrity constraints. 
Similarly to the argument made before for $\HT$, formulas under negation in both $\TEL$ and $\DEL$ can be evaluated in their classical counterpart (For details refer to Appendix A).
Therefore, an integrity constraint $\bot \leftarrow \varphi \equiv \neg \varphi$, appearing in an ASP program, where $\varphi$ is a temporal/dynamic formula, only needs to check the entitlement of $\varphi$ under $\LTL$ (respectively $\LDL$) semantics. 
With this observations we now proceed to formally introduce the logics of $\LTL$ and $\LDL$ for finite traces expressed as $\LTLf$ and $\LDLf$, treating them in the classical setting, thus leaving the HT logic behind. 
